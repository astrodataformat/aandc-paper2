%% This is file `elsarticle-template-2-harv.tex',
%%
%% Copyright 2009 Elsevier Ltd
%%
%% This file is part of the 'Elsarticle Bundle'.
%% ---------------------------------------------
%%
%% It may be distributed under the conditions of the LaTeX Project Public
%% License, either version 1.2 of this license or (at your option) any
%% later version.  The latest version of this license is in
%%    http://www.latex-project.org/lppl.txt
%% and version 1.2 or later is part of all distributions of LaTeX
%% version 1999/12/01 or later.
%%
%% The list of all files belonging to the 'Elsarticle Bundle' is
%% given in the file `manifest.txt'.
%%
%% Template article for Elsevier's document class `elsarticle'
%% with harvard style bibliographic references
%%
%% $Id: elsarticle-template-2-harv.tex 155 2009-10-08 05:35:05Z rishi $
%% $URL: http://lenova.river-valley.com/svn/elsbst/trunk/elsarticle-template-2-harv.tex $
%%

%%\documentclass[preprint,authoryear,12pt]{elsarticle}

%% Use the option review to obtain double line spacing
%% \documentclass[authoryear,preprint,review,12pt]{elsarticle}

%% Use the options 1p,twocolumn; 3p; 3p,twocolumn; 5p; or 5p,twocolumn
%% for a journal layout:

%% Astronomy & Computing uses 5p
%% \documentclass[final,authoryear,5p,times]{elsarticle}
\documentclass[final,authoryear,5p,times,twocolumn]{elsarticle}

%% if you use PostScript figures in your article
%% use the graphics package for simple commands
%% \usepackage{graphics}
%% or use the graphicx package for more complicated commands
% \usepackage{graphicx}
%% or use the epsfig package if you prefer to use the old commands
%% \usepackage{epsfig}

%% The amssymb package provides various useful mathematical symbols
% \usepackage{amssymb}
\usepackage{gensymb}
%% The amsthm package provides extended theorem environments
%% \usepackage{amsthm}

\usepackage[pdftex,pdfpagemode={UseOutlines},bookmarks,bookmarksopen,colorlinks,linkcolor={blue},citecolor={green},urlcolor={red}]{hyperref}
% \usepackage{hypernat}

\usepackage{breakurl}

%% The lineno packages adds line numbers. Start line numbering with
%% \begin{linenumbers}, end it with \end{linenumbers}. Or switch it on
%% for the whole article with \linenumbers after \end{frontmatter}.
%% \usepackage{lineno}

%% natbib.sty is loaded by default. However, natbib options can be
%% provided with \biboptions{...} command. Following options are
%% valid:

%%   round  -  round parentheses are used (default)
%%   square -  square brackets are used   [option]
%%   curly  -  curly braces are used      {option}
%%   angle  -  angle brackets are used    <option>
%%   semicolon  -  multiple citations separated by semi-colon (default)
%%   colon  - same as semicolon, an earlier confusion
%%   comma  -  separated by comma
%%   authoryear - selects author-year citations (default)
%%   numbers-  selects numerical citations
%%   super  -  numerical citations as superscripts
%%   sort   -  sorts multiple citations according to order in ref. list
%%   sort&compress   -  like sort, but also compresses numerical citations
%%   compress - compresses without sorting
%%   longnamesfirst  -  makes first citation full author list
%%
%% \biboptions{longnamesfirst,comma}

% \biboptions{}

\journal{Astronomy \& Computing}

%% Upright single quotes in verbatim fields make FITS header examples
%% much more readable
\usepackage{upquote}

%% For draft use color package to indicate open questions that need
%% clarification
\usepackage{color}

\defcitealias{2015Thommas}{Paper~I}

\begin{document}

\begin{frontmatter}

%% Title, authors and addresses

%% use the tnoteref command within \title for footnotes;
%% use the tnotetext command for the associated footnote;
%% use the fnref command within \author or \address for footnotes;
%% use the fntext command for the associated footnote;
%% use the corref command within \author for corresponding author footnotes;
%% use the cortext command for the associated footnote;
%% use the ead command for the email address,
%% and the form \ead[url] for the home page:
%%
%% \title{Title\tnoteref{label1}}
%% \tnotetext[label1]{}
%% \author{Name\corref{cor1}\fnref{label2}}
%% \ead{email address}
%% \ead[url]{home page}
%% \fntext[label2]{}
%% \cortext[cor1]{}
%% \address{Address\fnref{label3}}
%% \fntext[label3]{}

\title{Some Initial Requirements for Modern Astronomical Data Formats}

%% use optional labels to link authors explicitly to addresses:
%% \author[label1,label2]{<author name>}
%% \address[label1]{<address>}
%% \address[label2]{<address>}

\author[noao]{Brian~Thomas\corref{cor1}}
\ead{bthomas@noao.edu}
\author[cornell]{Tim~Jenness}
%\author[noao]{Frossie~Economou}
%\author[stsci]{Perry~Greenfield}
%\author[geminin]{Paul~Hirst}
%\author[jac]{David~S.~Berry}
%\author[stsci]{Erik~Bray}
%\author[glasgow]{Norman~Gray}
%\author[ohio]{Demitri~Muna}
%\author[geminis]{James~Turner}
%\author[princeton]{Miguel~de~Val-Borro}
%\author[iaa,ska]{Juande~Santander-Vela}
%\author[ipac]{David~Shupe}
%\author[ipac]{John~Good}
%\author[ipac]{G.~Bruce~Berriman}
%\author[icrar]{Slava~Kitaeff}
%\author[microsoft]{Jonathan~Fay}
%\author[sao]{Omar~Laurino}
%\author[ipac]{Walter~Landry}
%\author[nrao]{Joe~Masters}
%\author[cornell]{Adam~Brazier}
%\author[aifa]{Reinhold~Schaaf}
%\author[uwaterloo]{Kevin~Edwards}
%\author[jac]{Russell~O.~Redman}
%\author[warwick]{Thomas~R.~Marsh}
%\author[aip]{Ole~Streicher}
%\author[noao]{Pat~Norris}
%\author[ucm]{Sergio~Pascual}
%\author[unsw]{Matthew~Davie}
%\author[stsci]{Michael~Droettboom}
%\author[mpia]{Thomas~Robitaille}
%\author[iasf]{Riccardo~Campana}
%\author[psu]{Alex~Hagen}
%\author[mps]{Paul~Hartogh}
%\author[aifa]{Dominik~Klaes}
%\author[msum]{Matthew~W.~Craig}
%\author[cral]{Derek~Homeier}

%\cortext[cor1]{Corresponding author}

\address[noao]{Science Data Management, National Optical Astronomy Observatory, 950 N Cherry Ave, Tucson, AZ 85719, USA}
\address[cornell]{Department of Astronomy, Cornell University, Ithaca, NY 14853, USA}
%\address[stsci]{Space Telescope Science Institute, 3700 San Martin Drive, Baltimore, MD 21218, USA}
%\address[geminin]{Gemini Observatory, 670 N.\ A`oh\=ok\=u Place, Hilo, HI 96720, USA}
%\address[jac]{Joint Astronomy Centre, 660 N.\ A`oh\=ok\=u Place, Hilo, HI 96720, USA}
%\address[glasgow]{SUPA School of Physics \& Astronomy, University of Glasgow, Glasgow, G12 8QQ, UK}
%\address[ohio]{Department of Astronomy, The Ohio State University, Columbus, OH 43210, USA}
%\address[geminis]{Gemini Observatory, Casilla 603, La Serena, Chile}
%\address[princeton]{Department of Astrophysical Sciences, Princeton University, Princeton, NJ 08544, USA}
%\address[iaa]{Instituto de Astrof\'isica de Andaluc\'ia, Glorieta de la Astronom\'ia s/n, E-18008, Granada, Spain}
%\address[ska]{Square Kilometre Array Organisation, Jodrell Bank Observatory, Lower Withington, Macclesfield SK11~9DL, UK}
%\address[ipac]{Infrared Processing and Analysis Center, Caltech, Pasadena, CA 91125, USA}
%\address[icrar]{International Centre for Radio Astronomy Research, M468, 35 Stirling Hwy, Crawley, Perth WA 6009, Australia}
%\address[microsoft]{Microsoft Research, 14820 NE 36th Street, Redmond, WA 98052, USA}
%\address[sao]{Smithsonian Astrophysical Observatory, 60 Garden Street, Cambridge, MA 02138, USA}
%\address[nrao]{National Radio Astronomy Observatory, 520 Edgemont Road, Charlottesville, VA 22903, USA}
%\address[aifa]{Argelander-Institut f\"{u}r Astronomie, Universit\"{a}t Bonn, Auf dem H\"{u}gel 71, 53121 Bonn, Germany}
%\address[uwaterloo]{Department of Physics, University of Waterloo, Waterloo, ON N2L~3G1, Canada}
%\address[warwick]{Department of Physics, University of Warwick, Coventry CV4 7AL, UK}
%\address[aip]{Leibniz-Institut für Astrophysik Potsdam (AIP), An der Sternwarte 16, 14482 Potsdam, Germany}
%\address[ucm]{Departamento de Astrof\'{i}sica, Universidad Complutense de Madrid, 28040, Madrid, Spain}
%\address[unsw]{Department of Astrophysics, School of Physics, University of New South Wales, Sydney, NSW 2052, Australia}
%\address[mpia]{Max-Planck-Institut f\"{u}r Astronomie, K\"{o}nigstuhl 17, 69117 Heidelberg, Germany}
%\address[iasf]{Institute for Space Astrophysics and Cosmic Physics, Via Piero Gobetti 101, Bologna, I-40129, Italy}
%\address[psu]{Dept.\ of Astronomy and Astrophysics, The Pennsylvania State University, 525 Davey Lab, University Park, PA 16802, USA}
%\address[mps]{Max-Planck-Institut f\"{u}r Sonnensystemforschung, Justus-von-Liebig-Weg 3, 37077 G\"{o}ttingen, Germany}
%\address[msum]{Department of Physics and Astronomy, Minnesota State University Moorhead, 1104 7th Ave. S., Moorhead, MN 56563, USA}
%\address[cral]{Centre de Recherche Astrophysique de Lyon, UMR 5574, CNRS, Universit\'{e} de Lyon, ENS Lyon, %% \'{E}cole Normale Sup\'{e}rieure de Lyon, 46 All\'{e}e d'Italie, 69364 Lyon Cedex 07, France}

\begin{abstract}
%% Text of abstract

In \citetalias{2015Thommas} we pointed out that
the Flexible Image Transport System (FITS) standard, while a useful and
reliable shared dataformat in astronomy, is showing its age.
The format is limited in handling a subset of existing needs in the community.
Some example limitations include the need to handle an expanded range of 
specialized data product types (data models), being more conducive to the 
networked exchange and storage of data, handling very large datasets, and capturing
significantly more complex metadata and data relationships.

The community would suffer a very significant loss if we do not update our shared
standard to handle these limitations. Towards this end, this paper attempts to
capture requirements from a broad sampling of the astronomical community for 
shared astronomical data formats. 

In this paper we consider the requirements for ``archive", ``pipeline
processing", ``instrument capture"
and ``application" data formats in astronomy.

\end{abstract}

\begin{keyword}%% keywords here, in the form: keyword \sep keyword

%% MSC codes here, in the form: \MSC code \sep code
%% or \MSC[2008] code \sep code (2000 is the default)

FITS \sep
File formats \sep
Standards
\end{keyword}

\end{frontmatter}

% \linenumbers

\newcommand{\aspconf}{ASP Conf.\ Ser}
\newcommand{\aap}{A\&A}
\newcommand{\aaps}{A\&AS}
\newcommand{\jrasc}{JRASC}
\newcommand{\qjras}{QJRAS}
\newcommand{\mnras}{MNRAS}
\newcommand{\pasp}{PASP}
\newcommand{\pasa}{PASA}
\newcommand{\apjs}{ApJS}

%% main text
\section{Introduction}
\label{sec:intro}

Some of the limitations in the Flexible Image Transport System standard (FITS;
\citealt{1981A&AS...44..363W,1981A&AS...44..371G,2001A&A...376..359H,2010A&A...524A..42P}; and references therein)
have been highlighted in a previous work in this article series
\citep[][hereafter referred to as \citetalias{2015Thommas}]{2015Thommas}.
In this paper we wish to focus on describing the requirements of a modern
astronomical data format.

The paper is structured as follows: in Section~\ref{sec:process}, we describe
the process to capture requirements of the data format as identified by members
of the astronomy community with a diverse background and expertise.
Section~\ref{sec:results} compiles a list of essential requirements.

\section{Process for capturing requirements}
\label{sec:process}

\section{Results}
\label{sec:results}

\subsection{Requirement 1}
\label{sec:req_1}

\subsection{Requirement 2}
\label{sec:req_2}

\subsection{Requirement 3}
\label{sec:req_3}

\section{Summary}
\label{sec:summary}

\section{Acknowledgments}


The authors wish to thank ...


%% The Appendices part is started with the command \appendix;
%% appendix sections are then done as normal sections
%% \appendix

%% \section{}
%% \label{}

%% References
%%
%% Following citation commands can be used in the body text:
%%
%%  \citet{key}  ==>>  Jones et al. (1990)
%%  \citep{key}  ==>>  (Jones et al., 1990)
%%
%% Multiple citations as normal:
%% \citep{key1,key2}         ==>> (Jones et al., 1990; Smith, 1989)
%%                            or  (Jones et al., 1990, 1991)
%%                            or  (Jones et al., 1990a,b)
%% \cite{key} is the equivalent of \citet{key} in author-year mode
%%
%% Full author lists may be forced with \citet* or \citep*, e.g.
%%   \citep*{key}            ==>> (Jones, Baker, and Williams, 1990)
%%
%% Optional notes as:
%%   \citep[chap. 2]{key}    ==>> (Jones et al., 1990, chap. 2)
%%   \citep[e.g.,][]{key}    ==>> (e.g., Jones et al., 1990)
%%   \citep[see][pg. 34]{key}==>> (see Jones et al., 1990, pg. 34)
%%  (Note: in standard LaTeX, only one note is allowed, after the ref.
%%   Here, one note is like the standard, two make pre- and post-notes.)
%%
%%   \citealt{key}          ==>> Jones et al. 1990
%%   \citealt*{key}         ==>> Jones, Baker, and Williams 1990
%%   \citealp{key}          ==>> Jones et al., 1990
%%   \citealp*{key}         ==>> Jones, Baker, and Williams, 1990
%%
%% Additional citation possibilities
%%   \citeauthor{key}       ==>> Jones et al.
%%   \citeauthor*{key}      ==>> Jones, Baker, and Williams
%%   \citeyear{key}         ==>> 1990
%%   \citeyearpar{key}      ==>> (1990)
%%   \citetext{priv. comm.} ==>> (priv. comm.)
%%   \citenum{key}          ==>> 11 [non-superscripted]
%% Note: full author lists depends on whether the bib style supports them;
%%       if not, the abbreviated list is printed even when full requested.
%%
%% For names like della Robbia at the start of a sentence, use
%%   \Citet{dRob98}         ==>> Della Robbia (1998)
%%   \Citep{dRob98}         ==>> (Della Robbia, 1998)
%%   \Citeauthor{dRob98}    ==>> Della Robbia


%% References with bibTeX database:

\bibliographystyle{model2-names}
\bibliography{astrodataformat}

%% Authors are advised to submit their bibtex database files. They are
%% requested to list a bibtex style file in the manuscript if they do
%% not want to use model2-names.bst.

%% References without bibTeX database:

% \begin{thebibliography}{00}

%% \bibitem must have one of the following forms:
%%   \bibitem[Jones et al.(1990)]{key}...
%%   \bibitem[Jones et al.(1990)Jones, Baker, and Williams]{key}...
%%   \bibitem[Jones et al., 1990]{key}...
%%   \bibitem[\protect\citeauthoryear{Jones, Baker, and Williams}{Jones
%%       et al.}{1990}]{key}...
%%   \bibitem[\protect\citeauthoryear{Jones et al.}{1990}]{key}...
%%   \bibitem[\protect\astroncite{Jones et al.}{1990}]{key}...
%%   \bibitem[\protect\citename{Jones et al., }1990]{key}...
%%   \harvarditem[Jones et al.]{Jones, Baker, and Williams}{1990}{key}...
%%

% \bibitem[ ()]{}

% \end{thebibliography}

\end{document}

%%
%% End of file `elsarticle-template-2-harv.tex'.
